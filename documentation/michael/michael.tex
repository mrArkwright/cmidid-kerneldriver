\documentclass[paper=a4,fontsize=11pt,twocolumn,pagesize,bibtotoc]{scrartcl}

\usepackage[utf8]{inputenc}
\usepackage[T1]{fontenc}
\usepackage[english]{babel}

\usepackage{tikz}
\usepackage{amssymb,amsmath}
\usepackage{graphicx}
\usepackage{color}

\usepackage{listings}

% Customize the appearance of code listings:
\lstset{
  backgroundcolor=\color{white},
  basicstyle=\ttfamily,
  breakatwhitespace=false,
  breaklines=true,
  captionpos=b,
  commentstyle=\color{green},
  frame=single,
  keepspaces=true,
  keywordstyle=\color{blue},
  language=C,
  numbers=left,
  numbersep=5pt,
  numberstyle=\tiny\color{gray},
  rulecolor=\color{black},
  showspaces=false,
  showstringspaces=false,
  showtabs=false,
  stepnumber=2,
  stringstyle=\color{mauve},
  tabsize=2,
  title=\lstname
}

\usepackage{hyperref}

% Customize the appearance of hyperlinks and of the pdf reader:
\hypersetup{
  colorlinks=true, % removes border, allows text coloring
  linkcolor=blue, % color of in-document links.
  urlcolor=blue, % color for weblinks/email
  pdfborderstyle={/S/U/W 1} % 1pt underline (instead of a box)
}

% Remove the annoying ACM copyright notice:
\makeatletter
\def\@copyrightspace{\relax}
\makeatother

\title{CMIDID Kernel Driver\\Module Setup and GPIO Input}
\subtitle{Praktikum Linux \& C - Technische Universität München}

\author{Michael Opitz}

\begin{document}
\maketitle

\section{Introduction}
\label{michael:introduction}

In this part of the documentation, the setup of the CMIDID driver and the 
distribution into several components will be explained, as well as the 
implementation of the GPIO component, which is the one that handles the input 
on GPIO ports and translates that input into MIDI events.
I spent most of my time during the 
development phase working on the latter component, but before we actually 
started to implement the driver, I looked into the high-resolution timer API 
of the kernel. That was because we knew from the beginning that measuring the 
time difference between consecutive key hits (of our breadboard MIDI keyboard) 
would be required. Fortunately, the hrtimer API turned out to be very 
userfriendly and in the end we only required some simple functions like 
\begin{lstlisting}
ktime_get();
hrtimer_start(struct hrtimer *timer, ktime_t tim, const enum hrtimer_mode mode);
hrtimer_init(struct hrtimer *timer, clockid_t which_clock, enum hrtimer_mode mode);
\end{lstlisting}
and a couple more (see: \texttt{linux/include/linux/hrtimer.h}).
So, we decided that we don't need a separate module 
component to handle the timer related functionality and included this in the 
GPIO component.

So, besides the timer component, which was not required, and the GPIO 
component I already mentioned, we did add a main component, which handles 
initialization of the other components, as well as a MIDI component, which is 
responsible for creating a virtual MIDI device, and for outputting MIDI 
events on that device.

In the section \textbf{\nameref{michael:setup}}, I'm going to show how the compilation of 
the several module components works and how the component initialization and 
exit routines have to be handled. The other part of his documentation is 
a detailed explanation of the GPIO component: \textbf{\nameref{michael:gpios}}. This 
includes the steps necessary to read input on the GPIO ports from the kernel 
module, a description of a clean exit routine, as well as a couple of special 
cases and workaround for bugs, that are worth mentioning. The GPIO component 
has a significant amount of code for handling module customization via 
\texttt{ioctl} and module parameters; this won't be explained in this part of 
the documentation, but rather in (TODO: CITE JANNIK). And for an indepth guide 
on the MIDI component, take a look at (TODO: CITE ANDI).

\section{Project Setup}
\label{michael:setup}

The CMIDID kernel driver found in the \texttt{module} subdirectory, is split
up into the following components. The following enumeration and the following
subsections contain a short overview for each of those components as well as 
short explanation of our decision as to why we needed/created those components.
\begin{itemize}
  \item The main component: \texttt{cmidid\_main.c}. That's the one that 
    handles the actual module initialization and the module cleanup, aw well 
    as the creation of a character device for \texttt{ioctl}, and the 
    \texttt{ioctl} callback function. See section \textbf{\nameref{component:main}}.
  \item The MIDI component: \texttt{cmidid\_midi.c}. This component handles 
    the creation of the virtual MIDI soundcard via the ALSA sequencer 
    interface, and the sending of MIDI events (in our case only the events 
    \texttt{noteon} and \texttt{noteoff}). See section \textbf{\nameref{component:midi}}.
  \item The GPIO component: \texttt{cmidid\_gpio.c}. This component is 
    responsible for reading input on the GPIO ports, for translating the 
    input into MIDI keyboard key hit events and for informing the MIDI 
    component to actually send a MIDI event. See section \textbf{\nameref{component:gpio}}.
\end{itemize}
In-tree modules are usually contained in a signle \texttt{.c} file, but we 
used this setup to make a clean seperation of the individual logical parts of 
the module and to increase readability of each single source file. We added 
a header for every single component, where the \texttt{cmidid\_ioctl.h} header 
is special, in the way that it needs to be included by userspace programs that
want to communicate with the module via \texttt{ioctl}. And the \texttt{cmidid\_util.h}
header contains only a couple of \texttt{printk} macros to ease the output of 
debug information into the kernel log. The final structure of our project 
setup looks like this:
\begin{lstlisting}
module/
  Makefile
  cmidid_main.h
  cmiddi_main.c
  cmidid_gpio.h
  cmidid_gpio.c
  cmidid_midi.h
  cmidid_midi.c
  cmidid_ioctl.h
  cmidid_util.h
\end{lstlisting}
Additional to following overviews of each component, the next section provides
a brieft introduciton of the KBuild system and how we used this to compile 
the components into a single module.

\subsection{Compilation with KBuild}
\label{component:compilation}

The Linux kernel uses a system named KBuild which is thoroughly described in 
\url{https://github.com/torvalds/linux/blob/master/Documentation/kbuild/}.
For building external (i.e. out-of-tree) modules, the documentation file
\url{https://github.com/torvalds/linux/blob/master/Documentation/kbuild/modules.txt}
was especially relevant and explained the creation of makefiles for this type
of module.



\subsection{The Main Module Component}
\label{component:main}

\subsection{The MIDI Module Component}
\label{component:midi}

\subsection{The GPIO Component}
\label{component:gpio}

\section{Working with GPIOs}
\label{michael:gpios}



\end{document}

